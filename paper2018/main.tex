\def\year{2019}\relax
%File: formatting-instruction.tex
\documentclass[letterpaper]{article} %DO NOT CHANGE THIS
\usepackage{aaai19}  %Required
\usepackage{times}  %Required
\usepackage{helvet}  %Required
\usepackage{courier}  %Required
\usepackage{url}  %Required
\usepackage{graphicx}  %Required
\usepackage{subcaption}
\usepackage{multirow}

\frenchspacing  %Required
\setlength{\pdfpagewidth}{8.5in}  %Required
\setlength{\pdfpageheight}{11in}  %Required
%PDF Info Is Required:
  \pdfinfo{
/Title (Merging datasets through deep learning)
/Author (Kavitha Srinivas, Yehuda Gale, Julian Dolby)}
\setcounter{secnumdepth}{0}  
\usepackage{bm}

 \begin{document}
% The file aaai.sty is the style file for AAAI Press 
% proceedings, working notes, and technical reports.
%
\title{Merging Datasets Through Deep learning}


%\author{Kavitha Srinivas \\ IBM Research
%\And Yehuda Gale \\ Yeshiva University
%\And Julian Dolby \\ IBM Research}


\maketitle
\begin{abstract}
Merging datasets is a key operation for data analytics.  A frequent
requirement for merging is joining across columns that have
different surface forms for the same entity (e.g., the name of a
person might be represented as \textit{Douglas Adams},
\textit{Adams, Douglas} or \textit{Douglas Noel Adams}).  Similarly,
ontology alignment can require recognizing distinct surface forms of
the same entity, especially when ontologies are independently
developed.  However, data management systems are currently limited
to performing merges based on string equality, or at best using
string similarity.  We propose a novel approach to performing merges
based on deep learning models.  Our approach depends on (a) creating
a deep learning model that maps surface forms of an entity into a
set of vectors such that alternate forms for the same entity are
closest in vector space, (b) indexing these vectors using a nearest
neighbors algorithm to find the forms that can be potentially joined
together.  To build these models, we had to adapt techniques from
metric learning, due to the properties of entity names.  We
demonstrate sample selection and loss functions, which work for this
domain.  To evaluate our approach, we used Wikidata as ground truth
and built models for datasets with approximately 1.1M people's names
(200K identities) and 130K company names (70K identities).  We found
that vector embeddings allow for joins with precision of .91-.93 and
recall of .70-.81.  We make the models available to the community
for aligning people or companies across multiple datasets.  
\end{abstract}

\section{Introduction}

Merging datasets is a key operation for data analytics. A frequent requirement for merging is a join between two columns that frequently have different surface representations for the same entity (e.g., the name of a person might be represented as \textit{Douglas Adams} or \textit{Douglas Noel Adams}, as shown in Table~\ref{table-example}.  This problem occurs for many entity types such as people's names, company names, addresses, product descriptions, conference venues, or even people's faces.  Data management systems have however, largely focussed solely on equi-joins, where string or numeric equality determines which rows should be joined, because such joins are efficient.

Extensions in data management systems for handling `fuzzy' joins across different surface forms for the same entity typically use string similarity algorithms such as edit-distance, Jaro-Winkler and TF-IDF (e.g., \cite{Cohen2003}).  In applying fuzzy joins, a key problem is to avoid quadratic comparisons between pairs of strings drawn from the two columns.  A blocking strategy is used to reduce the number of pairs for fuzzy joins, such that only strings that share a common prefix, or suffix will be compared further.  String matching algorithms often do not work however, because transformations of entity names in perfectly valid ways can yield very different strings.  As shown in Table~\ref{table-example}, \textit{John Smith} is more similar to \textit{John Adams} than \textit{Adams, John}, but of course it is \textit{Adams, John} that is the valid alternate surface form for \textit{John Adams}.

More recently, data driven approaches have emerged as a powerful alternative to string matching techniques for fuzzy joins.  Data driven approaches mine patterns in the data to determine the `rules' for joining a given entity type.  One example of such an approach is illustrated in \cite{He:2015:SJS:2824032.2824036}, which determines which cell values should be joined based on whether those cell values co-occur on the same row across disparate tables in a very large corpus of data.  Another example of a data driven approach is work by \cite{auto-join-joining-tables-leveraging-transformations} where program synthesis techniques are used to learn the right set of transformations needed to perform the entity matching operation.  

In this paper, we propose a novel approach to the problem of joining different surface representations of the same entity, inspired in part by recent advances in deep learning.  Our approach depends on (a) creating a function that maps surface forms into a set of vectors such that alternate forms for the same entity are closest in vector space, (b) indexing these vectors using a space partitioning algorithm to find the forms that can be potentially joined together.  Our approach for merging datasets is much more flexible than prior data driven approaches (e.g., \cite{He:2015:SJS:2824032.2824036}, \cite{auto-join-joining-tables-leveraging-transformations}) because it can be applied to any data provided one can create a mapping function such that joinable surface forms of the entity are closer in vector space than disparate forms.  For instance, in face recognition and person re-identification, there has been a significant research effort aimed at creating deep learning models that learn to map different surface forms of a face closer together in a vector space.  In our proposed approach, we can now join two columns if they have alternate faces for the same entity, whereas in prior work, those faces needed to either have been joined before in some existing dataset, or the transforms of a face need to be discoverable as a program, which is near impossible for faces.  Our approach for merging datasets is also efficient because it eliminates quadratic comparisons of cell values in each column by using a mechanism that is itself data-driven.  Because we use space partitioning algorithms to find surface forms that are potentially joinable, the search process is efficiently eliminates large parts of the vector space from consideration for any given queried vector.  Across all cell values that are considered for a join, the efficiency of the search is $\mathcal{O}(n\log{}n)$.

Specifically, the first part of our solution to the join problem involves building a deep learning model that performs metric learning.  For the closely related problem of face recognition and person re-identification, deep neural networks have been used successfully to build models that map faces of the same entity closer in vector space estimate relative to faces from other entities.  To test the generality of our approach, we tried to build a deep learning model model for names of people and companies. As in face recognition, our system for metric learning is built by training a `triplet siamese network' to learn to map different surface forms of the same entity closer in vector space than different forms of the same entity.  Entity names have very different properties from faces, so we needed to adapt the approaches used in face recognition to the problem of matching entity names with a novel method to choose the sample for training, and novel loss functions as objectives for the metric learning problem.  At training, the network is given a triplet consisting of an `anchor' (e.g. \textit{Douglas Adams}), a positive element (e.g. \textit{Douglas Noel Adams}) and a negative element (e.g., John Adams), and it learns to produce a small distance estimate for the positive pair (close to 0), and a large distance estimate for the negative pair, using an objective function. A classification form of this function that simply learns whether two names belong to the same entity can be used to perform fuzzy joins, but then we would need a blocking strategy analogous to what has been used in the literature to identify the set of joinable pairs \cite{auto-join-joining-tables-leveraging-transformations}.  

Our observation is that one can actually exploit what the siamese network produces to eliminate this blocking step altogether.  Specifically, the last hidden layer of the siamese network is effectively a `vector embedding' that contains the critical features needed for computing the distance estimate.  If such models are pre-built for a given entity type, we can then use the models to produce vector embeddings for all cell values in the two columns to be joined in a single linear pass, and index just the second column values with a space partitioning algorithm that lays out vectors in n-dimensional space.  Then, for each cell value in the first column to be joined, querying the k-nearest neighbors should provide the set of values in the second column that can be joined. In our work, we used approximate nearest neighbor algorithms, which have been applied successfully to billions of items \cite{JDH17}.

\begin{table}[!htb]
    \caption{Example of a merge problem}
    \begin{subtable}{.5\linewidth}
      \centering
        \caption{Table A}
        \begin{tabular}{|l|l|}
          \hline
           Dept & Name \\
           \hline
           1    & Douglas Adams \\
           2    & John Adams \\
           3  & Douglas Bard \\
           \hline
        \end{tabular}
    \end{subtable}%
    \begin{subtable}{.5\linewidth}
      \centering
        \caption{Table B}
        \begin{tabular}{|l|l|}
          \hline
           Name & Salary \\
           \hline
           Douglas Noel Adams & 2000 \\
           Adams, John & 3000 \\
           John Smith & 3000 \\
           \hline
        \end{tabular}
    \end{subtable}
    \label{table-example}
\end{table}

Our key contributions are as follows:
\begin{itemize}
\item We outline a general approach to the problem of merging datasets for different surface forms of the same entity.  Our approach consists of building a deep neural network model for specific entity types and using a space partitioning algorithm to find the nearest neighbors to join.  To test the generality of our approach, we built triplet loss based siamese networks for ~200,000 people's names and ~70,000 company names respectively, drawn from Wikidata.  We show that such networks did learn to map surface forms of the same entity closer in vector space.  Specifically, if we 83\% and 79\% of variants of the entity name appeared in the top 20 neighbors for each entity.  Moreover, within the 20 neighbors, 77\% and 75\% of the positive names were closer to the anchor than negative names for people and companies respectively.  Our results suggest that deep learning models can be used successfully for data driven joins, assuming that such models are built for common entity types.
\item Training neural networks for deep metric learning is difficult because one cannot show quadratic combinations of the data to the system.  In face recognition, a number of techniques exist for triplet selection, and methods for selecting triples is an active area of research.  We developed a new technique for triplet selection for entity names because the techniques from face recognition do not generalize well to entity names.  We also developed a novel loss function to train the network to discriminate entity names from names of other entities.  We compare the results for this loss function with the ones that have been used for face recognition, and show that this loss function can perform much better than the existing functions for deep learning in this domain.
\item We also perform a complexity analysis of how such models can then be used to perform joins across different surface forms of the same entity efficiently, with a complexity of $\mathcal{O}(n\log{}n)$.  
\end{itemize} 

The rest of the paper is organized as follows: Section~\ref{siamese networks} describes how we adapted siamese networks to the problem of entity matching, specifically the problem of triplet selection for entity names, and the need for a new type of loss function to improve metric learning for this problem, section ~\ref{joins} describes how to use these models for efficient joins across datasets, section ~\ref{dataset} describes characteristics of the datasets we used for training, as well as cleansing and data augmentation techniques we used to generate data, section ~\ref{results} provides the experimental results, section~\ref{related work} covers related work, and section~\ref{conclusions} describes future work and conclusions.

\section{Related Work}
Extensions in data management systems for joins typically use string similarity algorithms such as edit-distance, Jaro-Winkler and TF-IDF; e.g., \cite{Cohen2003}.  String matching algorithms work poorly for merging the same entity because valid transformations of entity names can yield very different strings.  More recently, data driven approaches mine patterns to determine the `rules' for joining a given entity type.  One example is \cite{He:2015:SJS:2824032.2824036}, which determines which cell values should be joined based on whether those cell values co-occur on the same row across disparate tables in a very large corpus of data.  Another example is \cite{auto-join-joining-tables-leveraging-transformations} where program synthesis techniques are used to learn the right set of transformations needed to perform the entity matching operation.  Our approach for merging datasets is more general because the mapping function generalizes the set of transformations that are allowed across surface forms of an entity, even if they cannot be directly expressed as program transforms.

Joint embeddings have been applied recently to linking relational tuples for entity resolution \cite{Mudgal},\cite{Bordawekar18}.  Here, each model is specific to the database on which it was trained because the attributes of an entity vary among databases.  Our focus is on techniques for name matching, and we develop general purpose embedding models for merging alternate surface forms of key entities.  

Metric learning is a well studied problem in face recognition, e.g., \cite{DBLP:conf/cvpr/SchroffKP15}, with a rich literature in triplet mining techniques.  The closest approach to ours is the use of nearest neighbors algorithms for semi-hard triplet mining \cite{DBLP:journals/corr/KumarHC0D17}.  For semi-hard triplet mining, one cannot look at fixed neighborhood sizes in building triplets.  If all the positives are further away from the anchor than the negatives in a given neighborhood size of $k$, it means that $k$ needs to be expanded until a neighborhood size is found that has the right characteristics.  Our approach in including `hard negatives' means we can use a fixed $k$ to generate samples.  An additional benefit is that at least for certain types of datasets, we show that metric learning with hard negatives is more effective than semi-hard mining.

The study of loss function effectiveness in metric learning is also a rich literature, with two basic types of loss functions that have been proposed: (a) local loss functions such as triplet loss \cite{DBLP:conf/cvpr/SchroffKP15}, angular loss \cite{Zhang:2016:DML:3088616.3088665} and improved loss \cite{DBLP:journals/corr/abs-1708-01682}, and (b) loss functions that operate on a more global level across a batch of training examples \cite{NIPS2016_6200}, \cite{DBLP:conf/cvpr/SongXJS16}, \cite{songCVPR17}.   Since our triplet selection is global, rather than batch based, we did not see the value of using global loss functions.

\section{Network Architecture}
\label{architecture}
\begin{figure}
\includegraphics[width=1.0\linewidth]{triplet_siamese_network}
\caption{Architecture of the siamese network}
\label{siamese_nets}
\end{figure}

Figure~\ref{siamese_nets} illustrates the siamese network architecture \cite{DBLP:conf/cvpr/SchroffKP15} we used to build a system for merging data.  As stated earlier, input is initially triples of a name (\textit{anchor}), another name for the same entity (\textit{positive}) and a name of another entity (\textit{negative}).  During the tokenization process, we kept punctuation such as `-', `,', and `.' in the name because they are important signals in processing a name.  For the network, input vectors are computed for each entity assuming a maximum name length of 10 for each entity (i.e., no entity could have a name that spanned more than 10 tokens).  A 100 dimensional character embedding was computed for each token in the entity using pretrained character embeddings \cite{hashimoto-jmt:2017:EMNLP2017}, which resulted in a 100 x 10 character encoding for each name used in a triplet.  We used a character encoding rather than a word based encoding primarily because many names were missing from word based vector embeddings.

These three input vectors from each triplet are fed to three identical networks that share the same weights.  Weight sharing ensures that the networks learn the same mapping function for any given input vector.  In our implementation, each of the three networks had 4 stacked layers of 128 unit Gated Recurrent Units (or GRUs) to capture the sequential nature of the input.  For name and textual data, positional information is critical, so we used GRUs instead of the convolutional neural networks (CNNs) that have been traditionally used in metric learning for face and image recognition. GRUs are a type of recurrent network \cite{cho-al-emnlp14} where each hidden unit updates its weights at a specific step in the sequence $t$ based on the current input $x_t$ and the value of the hidden unit from the prior step $h_{t-1}$.  

%$r_t$ is a reset gate which determines whether the state from the previous step $h_{t-1}$ should be ignored. $z_t$ is an update gate which determines whether the current hidden state should be updated with the new hidden state $\tilde{h_t}$.  Equations~\ref{eq_1}-\ref{eq_4} that govern the update at $h_t$, as summarized by \cite{colah}.
%\begin{equation}
%z_t = \sigma(W_z . [h_{t-1}, x_t])
%@\label{eq_1}
%\end{equation}

%\begin{equation}
%r_t = \sigma(W_r . [h_{t-1}, x_t])
%\label{eq_2}
%\end{equation}

%\begin{equation}
%\tilde{h_t} = tanh (W . [r_t * h_{t-1}, x_t])
%\label{eq_3}
%\end{equation}

%\begin{equation}
%h_t = (1- z_t) * h_{t-1} + z_t * \tilde{h_t}
%\label{eq_4}
%\end{equation}

The output of the last layer shown in the Figure~\ref{siamese_nets} as a dark layer produces an output vector embedding for the inputs.  These are fed to two layers which compute a euclidean distance between the \textit{anchor} and the \textit{positive} elements of a triplet (\textit{positive distance}), and the \textit{anchor} and the \textit{negative} elements of a triplet (\textit{negative distance}).  Conceptually, there are two objectives in metric learning, one to minimize \textit{positive distances}, and the other to maximize \textit{negative distances}.  As described below, this dual objective can be achieved by different loss functions.  We restrict ourselves to a discussion of the some of the more popular loss functions that are local in nature (i.e., they only look at a single triple).  

\section{Loss functions}
\label{loss_functions}

Let $\mathbf{x}$ represent an embedding for an entity name, and $\bf{x_{a}}$, $\bf{x_{p}}$, $\bf{x_{n}}$ reflect the vector embeddings of the \textit{anchor}, \textit{positive} and \textit{negative}, respectively.  We investigated four different loss functions, three of which have been used in prior literature to explore their effectiveness for the entity metric learning problem.

\begin{figure*}[htb]
    \centering
    \begin{subfigure}[t]{0.23\textwidth}
        \centering
        \includegraphics[width=.9\linewidth]{schroff_triplet}
        \caption{Triplet loss}
        \label{schroff_loss}
    \end{subfigure}%
    ~ 
    \begin{subfigure}[t]{0.23\textwidth}
        \centering 
        \includegraphics[width=.9\linewidth]{modified_loss}
        \caption{Modified loss}
        \label{modified_loss}
    \end{subfigure}
    ~ 
    \begin{subfigure}[t]{0.23\textwidth}
        \centering 
        \includegraphics[width=.9\linewidth]{angular_loss}
        \caption{Angular loss}
        \label{angular_loss}
    \end{subfigure}
    \caption{Loss functions}
\end{figure*}

\subsection{Triplet loss}

For face recognition, Schroff et al. \cite{DBLP:conf/cvpr/SchroffKP15}
propose a triplet loss function where the \textit{positive distances}
for each triplet $i$ in the set of $N$ triplets is separated from
\textit{negative distances} by a margin of $\alpha$, as shown in
Figure~\ref{schroff_loss} with the arrow pushing toward $\alpha$.  For
each of $N$ triples, $l_{triple}$ reflects the loss for a given triple
as follows: 
\begin{equation}
  l_{triple} =  \left[\|\bf{x_a} - \bf{x_p}\|^2 - \|\bf{x_a} -\bf{x_n}\|^2 + \alpha \right]_+
\label{schroff}
\end{equation}
where
\begin{equation}
 \left[.\right]_{+} = max(0, .)
\end{equation}
and the loss function that is minimized across all $N$ triples is given by
\begin{equation}
 L = \sum_{i}^{N} l_{triple}
\end{equation}
Note that in this formulation, it is assumed that embedding is normalized so $\|\bf{x} \| = 1$ because this normalization is robust across variations in illumination and contrast.  The value of $\alpha$ in the original work is a hyper-parameter that \cite{DBLP:conf/cvpr/SchroffKP15} was set to 0.2.

\subsection{Improved Loss}

 An improvement over the triplet loss function is proposed by
 \cite{DBLP:conf/cvpr/SchroffKP15} for the recognition of faces in
 videos \cite{Zhang:2016:DML:3088616.3088665}.  Conceptually, this
 function that we refer to as `improved loss' in the paper considers
 the distances from the \textit{positive} element of the triplet to
 the \textit{negative} element, and tries to push that
 difference toward $\alpha$ as well as the distance from
 \textit{anchor} to the \textit{negative}.  We show this in
 Figure~\ref{modified_loss} with two push arrows.  In addition, it corrects the fact the  original triplet loss function has no constraints on how close the positive distance should be.  For instance, it is possible for the \textit{anchor} and \textit{positive} form a large cluster with a large intra-class distance. The equations that achieve these constraints are described below.  Equation~\ref{psi} tries to reduce intra-class distance by ensuring it is less than $\hat{\alpha}$.  Equation~\ref{phi} tries to maximize inter-class distance by ensuring that the distance from the \textit{anchor} and \textit{positive} to the negative are both taken into account.  Equation~\ref{improved_loss} balances inter-class constraints with intra-class constraints with the parameter $\lambda$. 

\begin{equation}
  \psi_{triple} = \|\bf{x_{a}} - \bf{x_{p}}\|^2 - \hat{\alpha}
\label{psi}
\end{equation}
\begin{equation}
  \phi_{triple} = \|\bf{x_{a}} - \bf{x_{p}}\|^2 - (\|\bf{x_{a}}- \bf{x_{n}}\|^2 + \|\bf{x_{p}} - \bf{x_{n}}\|^2) / 2  + \alpha
\label{phi}
\end{equation}
\begin{equation}
  l_{triple} = max(0, \phi) + \lambda * max(0, \psi)
\label{improved_loss}
\end{equation}
The parameter $\hat{\alpha}$ for equation~\ref{psi} is set to 0.1, and $\lambda$ is set to .02 in equation~\ref{improved_loss}, and $\alpha$ was set to 1 in the paper \cite{Zhang:2016:DML:3088616.3088665}.  As in Schroff et al. \cite{DBLP:conf/cvpr/SchroffKP15}, the embeddings are normalized to 1 although the actual paper does not make it clear.

\subsection{Angular Loss}

Wang et al. \cite{DBLP:journals/corr/abs-1708-01682} define a novel
angular loss function which is not based on pairwise distances, but
rather is based on the angles of the triangle formed by the
\textit{anchor}, \textit{positive} and the \textit{negative} triplet.
Conceptually, they point out that since the \textit{anchor} and the
\textit{positive} pairs belong to the same class, the angle formed by
the \textit{anchor}, \textit{negative}, and \textit{positive} elements
should be as small as possible within that triangle.  Their loss
function is an attempt to minimize this angle, as defined in the
equations below.  The rough idea is that moving the positive nearer
and the negative further away each reduce the angle, which we
illustrate in Figure~\ref{angular_loss} with the angle to shrink.
\begin{equation}
\bf{x_{c}} = (\bf{x_{a}} + \bf{x_{p}}) / 2
\end{equation}
\begin{equation}
l_{triplet} = \left[[\bf{x_{a}} - \bf{x_{p}}\|^2 - 4 \tan^2 \alpha \|\bf{x_{n}} - \bf{x_{c}}\|^2 \right])_{+}
\end{equation}

\subsection{Adapted Loss}

The inspiration for our loss is to separate the effect of the positive
and negative distances as much as possible.  Rather than subtracting
the negative distance from the positive one, we want to negate the
negtive distance and then add the two.  To approximate this, we
subtract the negative distance from a margin, and use 0 instead if
that difference is negative.  We then combine the squared distances as
usual, as how in Figure~\ref{adapted}.

\begin{equation}
  l_{triple} =  \|\bf{x_a} - \bf{x_p}\|^2 + \left[ \alpha - \|\bf{x_a} -\bf{x_n}\| \right]_+^2
\label{adapted}
\end{equation}

\section{Triplet Selection}
As discussed earlier, deep metric learning for joins is a difficult problem for neural networks to learn because it requires that the discrimination of each anchor from all the other hard negatives.  We describe a popular approach batch based approach from the face recognition literature first to contrast it with our mechanism for triplet selection.

\subsection{Batch Based Triple Selection}
\begin{figure}
\subfloat[Batch based\label{triplet_selection}]{%
  \includegraphics[width=0.5\linewidth]{triplet_selection}
}
~
\subfloat[ANN based\label{ANN_selection}]{%
  \includegraphics[width=0.5\linewidth]{ANN_selection}
}
\caption{triplet selection}
\label{triplet_selection}
\end{figure}

In Schroff et al.'s work~\cite{DBLP:conf/cvpr/SchroffKP15} they described a mini-batch based triplet selection mechanism for training that has dominated the literature.  Conceptually, sampling the right triplets for fast network learning requires sampling a set of \textit{hard positives} and a set of \textit{hard negatives}, where a hard positive is defined as $argmax^i \| x_{a} - x_{p}^i \|^2$, where $i$ ranges over all $x_p$ and a hard negative is defined as $argmin^i \| x_{a} - x{_n}^i \|^2$, where $i$ ranges over all $x_n$. 

However, it is infeasible to compute these values for the entire dataset.  Calculating hard positives is easy because the number of \textit{positives} is small normally.  Calculating hard negatives is not possible for all but small datasets.  As a result, the triplets can be generated by a mechanism illustrated in Figure~\ref{triplet_selection}.  Instead of focusing on finding \textit{hard positives}, they instead pair every possible positive in the sample shown in the right panel in the figure with selected negatives, since the set of positives is usually quite small.  Furthermore, for negative examples, they try to select \textit{semi-hard negatives} instead of \textit{hard negatives}, where a \textit{semi-hard negative} has the property that $\|x_a - x_p \|^2 < \|x_a - x_n \|^2$ but by a margin smaller than $\alpha$, as shown in figure ~\ref{triplet_selection}.  

%As shown in the figure~\ref{triplet_selection} a \textit{semi-hard negative} is further from the \textit{anchor} than the \textit{positive}, but is not distant enough.  That is, the loss function still needs to move it closer to the margin $\alpha$ that is used in the triplet-loss function.  Schroff et al. further argue selecting \textit{hard negatives} such as $C$ in the figure may in fact lead to bad local minima early on in training, and lead to a collapsed model where the function always learns to return a value of 0, regardless of the pair being considered.

\subsection{Metric Based Triplet Selection}
For the problem of joins, we ideally want the anchor and all of its positives to be clustered closest together and separated from the nearest negatives as clearly as possible.  Approximate nearest neighbor (ANN) indexes are highly efficient methods for selecting the top-K neighbors of a given vector by Euclidean distance, cosine similarity or other distance metrics.  They are based on space partitioning algorithms, such as \textit{k-d trees}, where the vector space is iteratively bisected into two disjoint regions.  The average complexity to query the vectors of a neighbor is $O(log N)$ where N is the number of vectors in the dataset.  Implementations exist now for fast, memory-efficient ANN indexes that scale up to a billion vectors \cite{JDH17} using techniques to compress vectors in memory efficiently.  In our work, we used the Annoy ANN implementation\footnote{\url{https://github.com/spotify/annoy}} in our work which is based on the refinements to \textit{k-d trees} ideas described in \cite{ann_paper}.

Assuming one has the entire dataset indexed in an ANN index, the problem of triplet selection can be simplified by asking the ANN index for the top k-nearest neighbors of an \textit{anchor}, where k is given by the number of triplets that one desires to generate for each anchor.  As in earlier work, selection of \textit{hard positives} is not relevant because all positive data should be used to teach the network the right function.  Selection of negatives is the set of all nearest neighbors that are \textit{negatives}.  There is no explicit attempt to filter out \textit{hard negatives} in the approach.  The overall idea is that the set of negatives that appear in the nearest neighbor set at input are in fact the most important elements for the network to learn to discriminate from positives for a join.  Focusing on these elements, regardless of whether they are \textit{hard} or \textit{semi-hard} should lead to better discrimination for joins.  An ANN-based strategy also provides an important baseline to assess what if any learning was performed by the neural network in mapping input vectors to a different space.


\section{Applying deep learning models to joins}
\label{join_system}

\begin{figure*}[htb]
    \centering
    \begin{subfigure}[t]{0.24\linewidth}
        \centering
        \includegraphics[width=.9\linewidth]{join1}
        \caption{Columns to be merged}
        \label{join1}
    \end{subfigure}%
    ~ 
    \begin{subfigure}[t]{0.24\linewidth}
        \centering 
        \includegraphics[width=.9\linewidth]{join2}
        \caption{Create embeddings for each cell value in each column using the siamese model}
        \label{join2}
    \end{subfigure}
    ~ 
    \begin{subfigure}[t]{0.24\linewidth}
        \centering 
        \includegraphics[width=.9\linewidth]{join3}
        \caption{Index embeddings for left table's cell values in a nearest neighbors index}
        \label{join3}
    \end{subfigure}
    ~ 
    \begin{subfigure}[t]{0.24\linewidth}
        \centering 
        \includegraphics[width=.9\linewidth]{join4}
        \caption{For each cell in right table, find closest neighbor(s) and merge these rows}
        \label{join4}
    \end{subfigure}
    \caption{Overview of joins using deep learning}
    \label{join_fig}
\end{figure*}


Figure~\ref{join_fig} provides an overview of how deep learning models can be used for merging datasets.  Figure~\ref{join1} in the figure shows the two columns in the two datasets to be joined.  The datasets are labeled arbitrarily as a left table and a right table.  For each cell value in the two columns to be joined, we obtain vector embeddings from the siamese network that was used to estimate distance for same and different surface forms for the same entity.  Note that although the siamese network has three separate networks, each network within a triplet network is in fact identical to the other two networks because they share weights.  A single version of the network with learnt weights is used to generate the embeddings for the cell value as shown in figure ~\ref{join2}. For the left table cell values, vector embeddings are inserted into an approximate nearest neighbors index, as shown in figure ~\ref{join3}.  For each cell value in the right table, vector embeddings are used as `query vectors' to query the approximate nearest neighbors index as shown in figure ~\ref{join4}.  In our context, merging the datasets would involve joining the top $k$ rows in the left table that are `closest' in distance to each cell value in the right table.  Note that the choice of $k$ clearly has a direct effect on the tradeoff between precision and recall, but this is true for any type of join algorithm that is not based on equality.


\section{Benchmarks}
\label{datasets}
Our benchmarks were derived from Wikidata.  Specifically, we used the \textit{also known as} property from Wikidata to get alternate forms for the same entity name for people as well as companies.  For company names, we augmented the names and surface forms in Wikidata with data from the SEC\footnote{\url{https://www.sec.gov/dera/data/financial-statement-and-notes-data-set.html}}, which has former and more recent names for companies.  There were 213,106 names for people from the specific dump we extracted, and 70,946 names for companies.  The extracted files and the cleansing code are available on our repository. The extracted files however contained significant noise that we cleaned up programmatically.  We describe the cleansing rules for people and for companies separately because they were somewhat different.  In the case of people's names, we also augmented the data so the system could learn some common rules that define variants of a person's name.  This was not possible for company names.

\subsection{Cleansing people's names}
Wikidata has a number of historical figures which are not really names of people (e.g. Queen Elizabeth, Pope Leo).  If we detected a title in the name referring to royalty or qualifiers or Roman numerals which strongly indicated royalty, we dropped the name.  We also removed punctuations such as '...', and anything that was placed in parenthesese because they were not part of the name (e.g. a qualification such as the son of Jacob might appear in parentheses after a name).  Although we got the extract for English, there were frequently names in Chinese, Korean, Cyrillic, etc.  We removed these and restricted ourselves to names in ISO-8859-1 unicode.  All punctuations such as ',', '-', '.' etc were retained for names because they are strong indicators of how a name needs to be parsed.  People in wikidata have the main name for the entity, with aliases for the person specified in a different property.  We made sure that every alternate form had at least one name part in common with the main name to rule out 'nicknames' (e.g. `Father of the Nation' for George Washington).  We also dropped cases where the last name of a person was different (usually because a woman's name changed after marriage).      

\subsection{Cleansing company names}
As with people's names, we removed any text in parentheses because it usually was a qualification (e.g., IBM (company)).  We restricted ourselves to unicode ISO-8859-1.  The SEC data had a lot of strange company names that could be described with the regex pattern T[0-9]+ or [0-9]+, and we dropped these.  We tried to ensure we included names that shared some subset of characters with the main name, ensuring we would not drop acronyms when possible.  The check for acronyms tested if any of the initial letters of each name part occurred in the name.

\subsection{Augmenting people's names}
In many cases, we had no alternate forms for a person's name even if we did have their main name.  We augmented the data with the following rules.  If the main name for the person in Wikidata had two parts, we created new source forms as follows: (a) Last Name , First Name, (b) First Name Initial . Last Name, and (c) Last Name , First Name Initial.

If the main name in Wikidata had three parts, we created these additional source forms in addition to the ones listed above: (a) First Name Middle Name Initial Last Name (b) Last Name , First Name Middle Name (c) Last Name , First Name Middle Initial .

After cleansing, if a name had no alternate surface forms, they were dropped.  This resulted in 40,555 company names with an average of 2.2 names each, and 195,422 people's names with an average of 4.69 names each.  Using the triplet selection algorithm we created a set of 10.28 million and 0.99 million triplets for people and companies at training.  The data was then split with a 60-20-20 ratio to provide training, validation and test data respectively.  Each run was conducted with a different random split to ensure generalization of results.

\subsection{Dataset characteristics}

\begin{wrapfigure}{r}{.65\textwidth}
    \subfloat[Recall rate as a function of neighborhood size\label{people_recall}]{% 
        \includegraphics[width=.49\linewidth]{people_recall}
    }
    ~ 
    \subfloat[Mean negative distance as a function of neighborhood size\label{people_distance}]{%
        \includegraphics[width=.49\linewidth]{people_distances}
     }
    \label{people_distances}
    \caption{Characteristics of people data}
\label{people_characteristics}
\end{wrapfigure}

 As a baseline, we measured how the anchors, positives and negatives were laid out in vector space based on character embeddings alone.  This gives us a measure of how difficult the problem is for the neural network to learn.  We indexed all the vector embeddings (regardless of test or train) into a nearest neighbors algorithm using the Spotify ANNOY\footnote{\url{https://github.com/spotify/annoy}} implementation.  We then queried the index for the $k$ nearest neighbors of this set, varying $k$ so it was either 20, 100, 500, or 1500 neighbors.  Our primary interest was in recall rates of positives prior to any training, to establish the baseline prior to training.  We also measured the nature of hard negatives as we varied the neighborhoods; i.e., what is the mean distance of negatives from the anchor while we increased neighborhoods, compared to the positive distances.  Figure \ref{people_characteristics} shows the results for the people data.  First, recall rate of positives in the nearest neighbor set was very low at 3\%, and it increased only to 6\% at a neighborhood size of 1500.  The difficulty of the problem for reconciling people's names is further highlighted by the distance data.  The mean distance of positives from anchor was 9.05, with a standard deviation of 3.08.  The mean distances of negatives from anchor was 2.73, with a standard deviation of 0.99 for $k$ of 20.  However even for $k$ of 1500, the mean negative distance was 3.64, well below the mean positive distance of 9.05.  The company data show a similar trend, although companies seems to be an easier problem than reconciling people data, as shown in Figure \ref{company_characteristics}.  Recall rate for companies starts at 16\% for a neighborhood size of 20, and is up to 25\% for a neighborhood size of 1500.  Mean negative distance is at 3.05 compared to 4.64 at a neighborhood size of 20, with a standard deviation of 1.44.  At a neighborhood size of 1500, mean negative sizes are slightly higher at 3.86. 

\begin{wrapfigure}{r}{.65\textwidth}
   \subfloat[Recall rate as a function of neighborhood size\label{company_recall}]{%
        \includegraphics[width=.49\linewidth]{company_recall}
   }
   ~
   \subfloat[Mean negative distance from anchor as a function of neighborhood size \label{company_distance}]{%
        \includegraphics[width=.49\linewidth]{company_distances}
   }
    \label{company_distances}
    \caption{Characteristics of company data}
\label{company_characteristics}
\end{wrapfigure}

\section{Experiments}
\label{results}
In building the models, we employed early stopping using the usual metric of accuracy of the validation data, and we performed hyper-parameter tuning using grid search varying margins for the adapted loss function from 1-20.  Accuracy was defined as the percentage of validation triples where positive distances from anchor were less than negative distances from anchor.  For all our runs, test accuracy as measured by this metric ranged in the 96-97\% range (CAN WE CONFIRM THIS ON OUR RUNS?).  Table \ref{Evaluation} shows the results for the people and company datasets, run with a fixed $k$ of 20 neighbors.  Because we compared across different losses, the results are categorized by each loss function for each dataset we tested.  For all the results reported here, we ran multiple runs because of the stochastic nature of neural network models; the results here are means across two runs. 

We report multiple metrics to measure the effectiveness of training, some of which are not standard because of the experimental setup, so we define them below:
\begin{itemize}
\item \textbf{Recall}.  Recall is measured by the percentage of positives in the nearest neighbor set of each anchor.
\item \textbf{Precision@1}.  Precision@1 is measured by the percentage of anchors with the very nearest neighbor being a positive.  As pointed out earlier, this is an important metric for assessing join performance in a majority of cases.
\item \textbf{Precision}.  Precision is measured by the percentage of all positives for each anchor that were closer to the anchor than any other negative relative to the size of the positive set.
\end{itemize}
 
\begin{table}[ht]
\caption{Precision and Recall by loss functions}
\label{precision_recall}
\begin{center}
%{\scriptsize
\begin{tabular}{|l|l|r|r|r|}
\hline
Entities & Loss & Recall & \multicolumn{2}{|c|}{Precision} \\
 & & & @1 & All \\
\hline
\multirow{4}{*}{People} & Adapted & .81 & .81 & .63 \\
\cline{2-5}
& Triplet & .74 & .84 & .55 \\
\cline{2-5}
& Improved & .71 & .52 & .45 \\
\cline{2-5}
& Angular & .04 & .09 & .02 \\
\hline
\multirow{4}{*}{Corps} & Adapted & .72 & .73 & .64 \\
\cline{2-5}
& Triplet & .75 & .76 & .67 \\
\cline{2-5}
& Improved & .74 & .72 & .64 \\
\cline{2-5}
& Angular & .26 & .32 & .22 \\
\hline
\end{tabular} %}
\end{center}
\end{table}


\begin{table}[ht]
\caption{Distance estimates after training}
\label{distance}
\begin{center}
%{\scriptsize 
\begin{tabular}{|l|l|r|r|r|r|}
\hline
Entities & Loss & \multicolumn{2}{|c|}{Pos} & \multicolumn{2}{|c|}{Neg} \\
& & Mean & Std & Mean & Std \\
\hline
\multirow{4}{*}{People} & Adapted & 1.59 & 4.04 & 2.44 & 2.45 \\
\cline{2-6}
& Triplet & .46 & .18 & .55 & .10 \\
\cline{2-6}
& Improved & .15 & .38 & .21 & .18 \\
\cline{2-6}
& Angular & .51 & .25 & .06 & .02 \\
\hline
\multirow{4}{*}{Corps} & Adapted & 2.38 & 3.77 & 3.23 & 1.71 \\
\cline{2-6}
& Triplet & .39 & .28 & .55 & .09 \\
\cline{2-6}
& Improved & .32 & .50 & .42 & .19 \\
\cline{2-6}
& Angular & .07 & .04 & .04 & .01 \\
\hline
\end{tabular}%}
\end{center}
\end{table}


\subsection{Performance for Joins}
For fuzzy joins, we need both precision and recall to be high.  Without good recall, a join will potentially miss names that should be joined.  Without high precision, a join will mistakenly join many inappropriate names.  To quantify recall, we measured, for each loss function.  The numbers for our adapted loss function were 81\% for people and 72\% for companies, so a join could capture most similar names.  Furthermore, the adapted loss function showed better performance on the harder people dataset, but there was a small advantage for the triplet loss function in the easier company data, see Table \ref{precision_recall}.  Furthermore, it appears that triplet loss and its variants are better than angular loss for learning this problem.

 Precision is a little trickier to define, but one way is to measure it is to examine how many true matches we get in the neighborhood of each anchor before seeing a single mistake, i.e. a match that should not be there.  That metric gives a picture of how many names would be correctly found, on average, by a join.  Our loss function gives 62\% by this metric for people and 64\% for companies, so about two thirds of recalled names would be found before finding a single error.  Since every name has at least one other name for the same entity, we also measure precision at 1, which is the probability that the very nearest neighbor is a true match.  For that, our loss yields 81\% for people and 73\% for companies.  Note that measuring precision for larger neighborhoods is tricky, since not every name has more than one name for the same entity (JULIAN I DONT UNDERSTAND WHAT THIS MEANS).


\subsection{Learning Performance}

 We also assessed our learning mechanism more directly by examining how the nearest neighborhood changed from before to after training.  As can be seen from Table \ref{distance}, recall is improved greatly, with the bigger change for people, in which case it improves from 3\% to 80\%.  For companies, it is from 17\% to 72\%.  Thus training is clearly effective in moving actual names for the same entity into the nearest neighbors.  For precision, the fraction of true matches found before the first error improves from 16\% to 73\% for people and 16\% to 64\% for companies.  Precision at one improves from 10\% to 81\% for people and from 26\% to 73\% for companies.  In both these cases, demonstrable training occurred.

\subsection{Comparison with Semi-Hard Negatives}
We hypothesized that training against hard negatives can potentially benefit on datasets with characteristics like those of our people dataset, when compared to training against semi-hard negatives.  Such datasets have large numbers of hard negatives, as suggested by Figure~\ref{people_characteristics}, which shows positive distances higher on average than negative ones.  
%Our strategy of focusing on the hardest negatives, i.e. those in the nearest neighborhood, ought to learn a nearest neighborhood with fewer negatives.  The common technique of using semi-hard negatives will focus more on negatives that are further than already-distant positives, and hence place less emphasis on the nearest neighborhood.

We therefore compared the hard negative triplet selection mechanism directly to training against semi-hard negatives.  To compute semi-hard negatives, we took, for each entity, all its positives, and found all negatives in the nearest neighborhood of that positive that were further from the entity than is the positive.  We made triplets for each such pair of positive and negative.  We thus chose the hardest semi-hard triplets: the negatives are as close to the positive as possible while still being further from the entity.  We ran experiments again for adapted loss on our people dataset, changing only the training used; these results are in Table~\ref{hard-semi-hard}.

\begin{table}[ht]
\caption{Precision and Recall for hard and semi-hard training}
\label{hard-semi-hard}
\begin{center}
%{\scriptsize
\begin{tabular}{|l|r|r|r|}
\hline
Training & Recall & Precision@1 & Precision \\
\hline
hard & .81 & .81 & .63 \\
\hline
semi-hard & .63 & .61 & .43 \\
\hline
\end{tabular}%}
\end{center}
\end{table}

 The results show training on hard negatives produces consistently better results, both for precision and recall.  Hard negative training resulted in recall of 81\% of posiives in the nearest set versus 63\% for semi-hard negatives.  Precision@1 is similar, with 81\% for hard negatives versus 61\% for semi-hard.  Overall precision, defined as the fraction of positives closer than any negative is 63\% for hard negatives versus 43\% for semi hard.

\subsection{Generalization Test on Faces}
We have demonstrated that our strategy for joins could work well for textual names of people and companies, but the technique could potentially work for any kind of data for which a vector embedding can be made.  To test how well that works, we evaluated two existing models for face recognition that were trained with the same approaches defined in the \cite{DBLP:conf/cvpr/SchroffKP15} on two different datasets, VGGFace2 \cite{DBLP:conf/fgr/CaoSXPZ18}, and CasiaWebFace \cite{DBLP:conf/cvpr/SchroffKP15}.  We took the two open sourced models \footnote{\url{https://github.com/davidsandberg/facenet}}, and extracted the output embeddings for faces from the LFW test set \cite{Huang2012a} using these embeddings.  We put these output embeddings into an ANN structure, and computed our metrics on that.  Recall of the same face was remarkably similar to our results for names, getting 80\% on the VGGFace2 dataset, and 76\% for the Casia Web Face dataset, as was precision (62\% for Casia Web faces, 68\% for VGGFace2).  However, we found precision at one was much lower on both datasets at 23\% for VGGFace2 and 21\% for Casia Web faces.

Our results contrasting hard versus semi-hard training suggest that one reason for lower precision at one could be the nature of the training.  However, the dataset is sufficiently different that this hypothesis that needs to be tested further empirically.

\subsection{Conclusion and Future Work}
We have described techniques that are useful to build deep learning models that can be used effectively for joins, and we will provide these models to the community for use.  Being able to build generic models that can be used for merging datasets seems to be a plausible idea, with caveats in terms of how one builds these models.

%Our results and our comparison with faces seems to show that training on hard negatives does not compromise recall overall and can boost precision for near neighbors.  Since joins will be most effective and efficient by focusing on a small, precise neighborhood, training on hard negatives may in fact be a useful addition.


%References and End of Paper
  %These lines must be placed at the end of your paper
  \bibliography{paper}
  \bibliographystyle{aaai}
  \end{document}

