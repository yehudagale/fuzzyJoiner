%\section{Applying deep learning models to joins}
%\label{join_system}

%\begin{figure*}[htb]
%    \centering
%    \begin{subfigure}[t]{0.24\linewidth}
%        \centering
%        \includegraphics[width=.9\linewidth]{join1}
%        \caption{Columns to be merged}
%        \label{join1}
%    \end{subfigure}%
%    ~ 
%    \begin{subfigure}[t]{0.24\linewidth}
%        \centering 
%        \includegraphics[width=.9\linewidth]{join2}
%        \caption{Create embeddings for each cell value in each column using the siamese model}
%        \label{join2}
%    \end{subfigure}
%    ~ 
%    \begin{subfigure}[t]{0.24\linewidth}
%        \centering 
%        \includegraphics[width=.9\linewidth]{join3}
%        \caption{Index embeddings for left table's cell values in a nearest neighbors index}
%        \label{join3}
%    \end{subfigure}
%    ~ 
%    \begin{subfigure}[t]{0.24\linewidth}
%        \centering 
%        \includegraphics[width=.9\linewidth]{join4}
%        \caption{For each cell in right table, find closest neighbor(s) and merge these rows}
%        \label{join4}
%    \end{subfigure}
%    \caption{Overview of joins using deep learning}
%    \label{join_fig}
%\end{figure*}

Assuming we have deep learning models that are trained to produce the right distance estimates for alternate surface forms for an entity, the models can be used for a join as follows.  For each cell value in the two columns to be joined, obtain vector embeddings from the last layer of the network.  Note that although the siamese network has three separate networks, each network is in fact identical to the other two networks because they share weights.  For the column cell values, vector embeddings are inserted into an approximate nearest neighbors index.  For each cell value in the right column, vector embeddings are used as `query vectors' to query the approximate nearest neighbors index.  In our context, merging the datasets would involve joining the top $k$ rows in the left table that are `closest' in distance to each cell value in the right table.  Note that the choice of $k$ clearly has a direct effect on the tradeoff between precision and recall, but for most practical uses of join, $k$ is usually very small (typically 1).  This has implications on what metrics we can use to evaluate joins, as we describe in section \ref{results}.

